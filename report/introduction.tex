\section{Introduction}
An interesting topic of research, in computer graphics is the detection of collisions and transformation of models, which are colliding. Models are usually represented by a tetrahedral mesh. These models have a large amount of tetrahedral elements and vertices, so GPUs are very useful for doing the calculations faster.\\
We will be focusing on collision detection, thus calculating which elements collide with each other. The most simple method would be to just check each pair of elements, and see wheter they collide. However, due to the large amount of elements, this will take to much time. A faster method uses a hierarchy of axis-aligned bounding boxes. This Bounding Volume Hierarchy (BVH) is represented by a binary tree. The leaves of the tree are the bounding boxes of the actual geometry (triangular mesh faces), and the nodes at different tree levels represent smallest-enclosing, axis-aligned bounding boxes of the bounding boxes of their child nodes. When two higher level bounding boxes do not collide, we can be sure that neither the elements in the bounding boxes will collide.\\

